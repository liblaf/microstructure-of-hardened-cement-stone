% !TeX root = main.tex

\section{实验建议}

本实验使用背散射电子图像 (BSE) 来区别水泥石中的各组分, 但是这种方法的有效性和必要性有待商榷.
因为根据扫描电子显微镜 (SEM) 图像, 我们可以根据各组分的形状和结构来区分它们, 而且利用能谱可以对各组分进行更精确的分析.
因此, 下次实验可以考虑不使用 BSE, 而是直接使用 SEM 和能谱来观察和分析水泥石中的各组分.

另外, 本实验使用的水泥石仅进行了 \SI{7}{\day} 的培养, 未能达到完全硬化的状态.
虽然这不影响本实验的目的, 但是为了更好地观察水泥石的微观结构, 下次实验可以考虑使用更长时间的培养, 以达到更好的硬化程度.
因为随着培养时间的延长, 水泥石中会发生一些化学反应, 如火山灰反应, 这会影响水泥石中的组分和微观结构.

\section{课程建议}

\wei{课程建议部分, 记得补充. }

首先, 我们要感谢老师在本学期为我们讲授了这门精彩的课程, 让我们对建筑材料有了更深入的了解和认识.
在此, 我们想对课程和老师提出一些建议, 希望能够对今后的教学有所帮助.

\begin{itemize}
  \item 我们觉得授课内容可以适当减少一些, 留出更多的课内时间开展同学间的小组讨论以及课程实验, 实验报告的撰写等活动.
        这样可以让我们更加主动地参与到学习中, 提高我们的动手能力和合作能力, 也可以加深我们对理论知识的理解和应用.
  \item 我们建议 PPT 的结构可以适当调整一下, 做成适合查阅的形式, 以适合作为学习的参考, 方便我们回顾和复习.
        另外, 也可以在 PPT 中加入一些关键词或重点提示, 帮助我们抓住重点和难点.
  \item 我们还希望课堂内容可以增加一些实际的例子和有趣的小故事, 以便同学更好地理解课程内容, 了解真实生产中的情况.
        我们觉得这样可以增加课堂的趣味性和互动性, 激发我们的学习兴趣和好奇心, 也可以拓宽我们的视野和知识面.
\end{itemize}

以上是我们对本门课程和老师的一些建议, 仅供参考.
再次感谢老师和助教的辛勤付出和耐心指导!
