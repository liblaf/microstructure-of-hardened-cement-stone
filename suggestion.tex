% !TeX root = main.tex

\section{实验建议}

本实验使用背散射电子图像 (BSE) 来区别水泥石中的各组分, 但是这种方法的有效性和必要性有待商榷.
因为根据扫描电子显微镜 (SEM) 图像, 我们可以根据各组分的形状和结构来区分它们, 而且利用能谱可以对各组分进行更精确的分析.
因此, 下次实验可以考虑不使用 BSE, 而是可以考虑仅使用 SEM 和能谱来观察和分析水泥石中的各组分.

另外, 本实验使用的水泥石仅进行了 \SI{7}{\day} 的养护, 未能达到完全硬化的状态.
虽然这不影响本实验的目的, 但是为了更充分地观察不同时期的水泥石的微观结构, 可以考虑拉长实验周期. 这是有意义的, 因为在第~\ref{sec:fa}~节中, 我们指出, 随着养护时间的延长, 水泥石中不断进行的火山灰反应会影响水泥石中粉煤灰的微观结构. 这样更改设计可以使得实验参与者理解火山灰反应的进行程度与时间的关系.

最后, 本次实验由于各类因素 (例如仪器较为精密等) 导致示范的部分较多而动手的部分相对较少.
建议可以在实验设计中区分示范与操作过程, 使得学生能更充分地体验实验操作.

\section{课程建议}

\wei{课程建议部分, 记得补充. }

首先, 我们要感谢老师在本学期为我们讲授了这门精彩的课程, 让我们对建筑材料有了更深入的了解和认识.
在此, 我们想对课程和老师提出一些建议, 希望能够对今后的教学有所帮助.

\begin{itemize}
  \item 我们觉得授课内容可以适当减少一些, 留出更多的课内时间开展同学间的小组讨论以及课程实验, 实验报告的撰写等活动.
        这样可以让我们更加主动地参与到学习中, 提高我们的动手能力和合作能力, 也可以加深我们对理论知识的理解和应用.
  \item 我们建议 PPT 的结构可以适当调整一下, 做成适合查阅的形式, 以适合作为学习的参考, 方便我们回顾和复习.
        另外, 也可以在 PPT 中加入一些关键词或重点提示, 帮助我们抓住重点和难点.
  \item 我们还希望课堂内容可以增加一些实际的例子和有趣的小故事, 以便同学更好地理解课程内容, 了解真实生产中的情况.
        我们觉得这样可以增加课堂的趣味性和互动性, 激发我们的学习兴趣和好奇心, 也可以拓宽我们的视野和知识面.
\end{itemize}

以上是我们对本门课程和老师的一些建议, 仅供参考.
再次感谢老师和助教的辛勤付出和耐心指导!

\section{致谢}

在本学期的《建筑材料》课程中, 我们收获了很多知识和技能, 感受到了建筑材料学科的魅力和挑战.
在此, 我们要向本课程的任课老师和各位助教表示衷心的感谢.

首先, 我们要感谢孔老师的精彩教学和悉心指导.
老师的课堂生动有趣, 深入浅出, 让我们对建筑材料的基本概念和性能有了清晰的认识.
老师还经常给予我们及时的反馈和鼓励, 帮助我们克服了学习中遇到的困难和挫折.

其次, 我们要感谢各位助教的辛勤付出和无私帮助.
王晶助教的实验设计合理有效, 培养了我们对建筑材料的分析和应用能力.
王助教和张尚枫学长在实验室上为我们提供了丰富的实践机会和资源, 让我们亲身体验了建筑材料的制备和测试过程.
王助教和张学长还耐心地解答了我们的各种疑问和困惑, 给了我们很多建议.

最后, 我们要感谢我们的同学们的友好合作和相互支持.
同学们在小组讨论和实验合作中展现了积极主动和团队精神, 与我们分享了各自的见解和经验, 共同创造了良好的学习氛围和动力.

在《建筑材料》课程中, 我们不仅学到了知识, 还收获了友情.
我们将珍惜这段美好的学习时光, 并继续努力提高自己.
再次感谢老师和助教们对我们的教育和培养!
