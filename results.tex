% !TeX root = main.tex

\section{实验结果}

\subsection{实验理论}
本实验的理论内容如图~\ref{fig:mechanism}所示, 硬化水泥石中包含水泥水化产物\ce{Ca(OH)2}, \ce{CSH}, \ce{AFt}, 可能包含未水化的\ce{C2S}与\ce{C3S}. 在加入粉煤灰的水泥石中应该包含不参与反应的粉煤灰. 

\begin{figure}[!t]
  \centering
  \includegraphics[width = 0.6\linewidth]{figures/exp3/mechanism.png}
  \caption{硬化水泥石中包含的化学反应以及化学物质. 本图来源于孔祥明老师《建筑材料》课程课件. }
  \label{fig:mechanism}
\end{figure}

\subsection{微观结构, 能谱图, 元素组成及成分分析}

\subsubsection{未水化的\ce{C3S}}
\begin{minipage}{\textwidth}
  \begin{minipage}[b]{0.32\textwidth}
    \centering
    \includegraphics[width = \linewidth]{assets/spectrum/00-01-10000x-ETD-C3S.png}
    \captionof{figure}{粉煤灰掺量0\%, 01区域的能谱图}
  \end{minipage}
  \hfill
  \begin{minipage}[b]{0.32\textwidth}
    \centering
    \includegraphics[width = \linewidth]{assets/spectrum selection/00-01-10000x-ETD-C3S.png}
    \captionof{figure}{粉煤灰掺量0\%, 01区域的ETD图像及选择区域}
    \label{fig:00-01-select}
  \end{minipage}
  \hfill
  \begin{minipage}[b]{0.32\textwidth}
    \centering
    \begin{tabular}{|c|c|c|}
      \hline
      Element & Wt \%  & At \%  \\ \hline
      C K     & 03.23 & 06.95 \\ \hline
      O K     & 27.17 & 43.92 \\ \hline
      MgK     & 01.58 & 01.69 \\ \hline
      AlK     & 03.04 & 02.92 \\ \hline
      SiK     & 12.12 & 11.16 \\ \hline
      CaK     & 48.70 & 31.43 \\ \hline
      FeK     & 04.16 & 01.93 \\ \hline
    \end{tabular}
    \captionof{table}{粉煤灰掺量0\%, 01区域的元素组成}
    \label{tab:00-01}
  \end{minipage}
\end{minipage}



\begin{minipage}{\textwidth}
  \begin{minipage}[b]{0.32\textwidth}
    \centering
    \includegraphics[width = \linewidth]{assets/spectrum/00-02-10000x-ETD-C3S.png}
    \captionof{figure}{粉煤灰掺量0\%, 02区域的能谱图}
  \end{minipage}
  \hfill
  \begin{minipage}[b]{0.32\textwidth}
    \centering
    \includegraphics[width = \linewidth]{assets/spectrum selection/00-02-10000x-ETD-C3S.png}
    \captionof{figure}{粉煤灰掺量0\%, 02区域的ETD图像及选择区域}
    \label{fig:00-02-select}
  \end{minipage}
  \hfill
  \begin{minipage}[b]{0.32\textwidth}
    \centering
    \begin{tabular}{|c|c|c|}
    \hline
    Element & Wt \%  & At \%  \\ \hline
    C K     & 02.69 & 05.74 \\ \hline
    O K     & 28.78 & 46.18 \\ \hline
    MgK     & 00.96 & 01.01 \\ \hline
    AlK     & 00.42 & 00.40 \\ \hline
    SiK     & 13.88 & 12.69 \\ \hline
    CaK     & 52.52 & 33.63 \\ \hline
    FeK     & 00.75 & 00.35 \\ \hline
    \end{tabular}
    \captionof{table}{粉煤灰掺量0\%, 02区域的元素组成}
    \label{tab:00-02}
  \end{minipage}
\end{minipage}


在ETD图像中发现了如图~\ref{fig:00-01-select}与图~\ref{fig:00-02-select}的微观结构, 选择以上微观结构区域进行能谱分析, 结果分别如表~\ref{tab:00-01}与表~\ref{tab:00-02}所示. 在表格中, 我们发现\ce{Ca}与\ce{Si}的物质的量之比大约为3:1, 考虑钙, 硅的物质的量之比, 该区域成分应该是\ce{C3S}. 

\subsubsection{未水化的\ce{C2S} }

\begin{minipage}{\textwidth}
  \begin{minipage}[b]{0.32\textwidth}
    \centering
    \includegraphics[width = \linewidth]{assets/spectrum/00-03-15000x-ETD-C2S.png}
    \captionof{figure}{粉煤灰掺量0\%, 03区域的能谱图}
  \end{minipage}
  \hfill
  \begin{minipage}[b]{0.32\textwidth}
    \centering
    \includegraphics[width = \linewidth]{assets/spectrum selection/00-03-10000x-ETD-C2S.png}
    \captionof{figure}{粉煤灰掺量0\%, 03区域的ETD图像及选择区域}
    \label{fig:00-03-select}
  \end{minipage}
  \hfill
  \begin{minipage}[b]{0.32\textwidth}
    \centering
    \begin{tabular}{|c|c|c|}
      \hline
      Element & Wt \%  & At \%  \\ \hline
      C K     & 10.18 & 19.55 \\ \hline
      O K     & 28.70 & 41.35 \\ \hline
      MgK     & 00.45 & 00.43 \\ \hline
      AlK     & 00.94 & 00.80 \\ \hline
      SiK     & 14.96 & 12.28 \\ \hline
      CaK     & 43.80 & 25.19 \\ \hline
      FeK     & 00.96 & 00.40 \\ \hline
      \end{tabular}
    \captionof{table}{粉煤灰掺量0\%, 03区域的元素组成}
    \label{tab:00-03}
  \end{minipage}
\end{minipage}

在ETD图像中发现了如图~\ref{fig:00-03-select}的微观结构, 选择以上微观结构区域进行能谱分析, 结果分别如表~\ref{tab:00-03}所示. 在表格中, 我们发现\ce{Ca}与\ce{Si}的物质的量之比大约为2:1, 考虑钙, 硅的物质的量之比, 该区域成分应该是\ce{C2S}. 


\subsubsection{水化产物\ce{CH} 或 \ce{Ca(OH)2} }

\begin{minipage}{\textwidth}
  \begin{minipage}[b]{0.32\textwidth}
    \centering
    \includegraphics[width = \linewidth]{assets/spectrum/00-05-10000x-ETD-CH.png}
    \captionof{figure}{粉煤灰掺量0\%, 05区域的能谱图}
  \end{minipage}
  \hfill
  \begin{minipage}[b]{0.32\textwidth}
    \centering
    \includegraphics[width = \linewidth]{assets/spectrum selection/00-05-10000x-ETD-CH.png}
    \captionof{figure}{粉煤灰掺量0\%, 05区域的ETD图像及选择区域}
    \label{fig:00-05-select}
  \end{minipage}
  \hfill
  \begin{minipage}[b]{0.32\textwidth}
    \centering
    \begin{tabular}{|c|c|c|}
      \hline
      
      Element & Wt \%  & At \%  \\ \hline
      C K     & 02.24 & 05.37 \\ \hline
      O K     & 22.10 & 39.79 \\ \hline
      MgK     & 00.30 & 00.36 \\ \hline
      AlK     & 00.41 & 00.44 \\ \hline
      SiK     & 00.90 & 00.93 \\ \hline
      CaK     & 73.53 & 52.85 \\ \hline
      FeK     & 00.52 & 00.27 \\ \hline
      \end{tabular}
    \captionof{table}{粉煤灰掺量0\%, 05区域的元素组成}
    \label{tab:00-05}
  \end{minipage}
\end{minipage}


\begin{minipage}{\textwidth}
  \begin{minipage}[b]{0.32\textwidth}
    \centering
    \includegraphics[width = \linewidth]{assets/spectrum/00-07-10000x-ETD-CH.png}
    \captionof{figure}{粉煤灰掺量0\%, 07区域的能谱图}
  \end{minipage}
  \hfill
  \begin{minipage}[b]{0.32\textwidth}
    \centering
    \includegraphics[width = \linewidth]{assets/spectrum selection/00-07-02000x-ETD-CH.png}
    \captionof{figure}{粉煤灰掺量0\%, 07区域的ETD图像及选择区域}
    \label{fig:00-07-select}
  \end{minipage}
  \hfill
  \begin{minipage}[b]{0.32\textwidth}
    \centering
    \begin{tabular}{|c|c|c|}
      \hline
      Element & Wt \%  & At \%  \\ \hline
      C K     & 01.37 & 03.63 \\ \hline
      O K     & 14.85 & 29.56 \\ \hline
      MgK     & 00.11 & 00.15 \\ \hline
      AlK     & 00.34 & 00.40 \\ \hline
      SiK     & 00.86 & 00.98 \\ \hline
      CaK     & 81.36 & 64.65 \\ \hline
      FeK     & 01.09 & 00.62 \\ \hline
      \end{tabular}
    \captionof{table}{粉煤灰掺量0\%, 07区域的元素组成}
    \label{tab:00-07}
  \end{minipage}
\end{minipage}

在ETD图像中发现了如图~\ref{fig:00-05-select}和图~\ref{fig:00-07-select}的片层状微观结构, 选择以上微观结构区域进行能谱分析, 结果分别如表~\ref{tab:00-05}和表~\ref{tab:00-07}所示. 在表格中, 我们发现在表格中主要的元素组成为\ce{Ca}与\ce{O}, 而\ce{Si}等其他元素很少, 推测该区域成分是\ce{CH}, 即\ce{Ca(OH)2}. 

\subsubsection{水化产物\ce{CSH}}

\begin{minipage}{\textwidth}
  \begin{minipage}[b]{0.32\textwidth}
    \centering
    \includegraphics[width = \linewidth]{assets/spectrum/00-08-10000x-ETD-CSH.png}
    \captionof{figure}{粉煤灰掺量0\%, 08区域的能谱图}
  \end{minipage}
  \hfill
  \begin{minipage}[b]{0.32\textwidth}
    \centering
    \includegraphics[width = \linewidth]{assets/spectrum selection/00-08-02000x-ETD-CSH.png}
    \captionof{figure}{粉煤灰掺量0\%, 08区域的ETD图像及选择区域}
    \label{fig:00-08-select}
  \end{minipage}
  \hfill
  \begin{minipage}[b]{0.32\textwidth}
    \centering
    \begin{tabular}{|c|c|c|}
      \hline
      Element & Wt \%  & At \%  \\ \hline
      C K     & 08.04 & 14.52 \\ \hline
      O K     & 40.04 & 54.29 \\ \hline
      MgK     & 00.48 & 00.43 \\ \hline
      AlK     & 01.49 & 01.20 \\ \hline
      SiK     & 12.39 & 09.57 \\ \hline
      CaK     & 35.43 & 19.17 \\ \hline
      FeK     & 02.13 & 00.83 \\ \hline
      \end{tabular}
    \captionof{table}{粉煤灰掺量0\%, 08区域的元素组成}
    \label{tab:00-08}
  \end{minipage}
\end{minipage}


\begin{minipage}{\textwidth}
  \begin{minipage}[b]{0.32\textwidth}
    \centering
    \includegraphics[width = \linewidth]{assets/spectrum/30-01-30000x-ETD-CSH.png}
    \captionof{figure}{粉煤灰掺量30\%, 01区域的能谱图}
  \end{minipage}
  \hfill
  \begin{minipage}[b]{0.32\textwidth}
    \centering
    \includegraphics[width = \linewidth]{assets/spectrum selection/30-01-30000x-CSH.png}
    \captionof{figure}{粉煤灰掺量30\%, 01区域的ETD图像及选择区域}
    \label{fig:30-01-select}
  \end{minipage}
  \hfill
  \begin{minipage}[b]{0.32\textwidth}
    \centering
    \begin{tabular}{|c|c|c|}
    \hline
    Element & Wt \%  & At \%  \\ \hline
    C K     & 09.18 & 15.71 \\ \hline
    O K     & 40.78 & 52.38 \\ \hline
    MgK     & 00.43 & 00.36 \\ \hline
    AlK     & 08.57 & 06.53 \\ \hline
    SiK     & 19.82 & 14.50 \\ \hline
    CaK     & 18.75 & 09.61 \\ \hline
    FeK     & 02.47 & 00.91 \\ \hline
    \end{tabular}
    \captionof{table}{粉煤灰掺量30\%, 01区域的元素组成}
    \label{tab:30-01}
  \end{minipage}
\end{minipage}

在ETD图像中发现了如图~\ref{fig:00-08-select}和图~\ref{fig:30-01-select}的网络状微观结构, 选择以上微观结构区域进行能谱分析, 结果如表~\ref{tab:00-08}和表~\ref{tab:30-01}所示. 在图中, 我们发现该微观结构并没有规则的结构, 也不存在规律性, 因此排除其为晶体的可能性; 同时, 由于其体积尺寸超出\SI{10}{\micro\meter} , 因而推断其属于\ce{CSH}. 同时, 在表格中, 我们发现在表格中主要的元素组成为\ce{Ca}, \ce{Si} 与 \ce{O}, 且\ce{O} 元素成分极大, 推测该区域成分是\ce{CSH} 凝胶. 
% 在ETD图像中发现了如图~\ref{fig:30-01-select}的网络状微观结构, 选择以上微观结构区域进行能谱分析, 结果如表~\ref{tab:30-01}所示. 在图中, 我们发现该微观结构并没有规则的结构, 也不存在规律性, 因此排除其为晶体的可能性; 同时, 由于其体积尺寸超出\SI{10}{\micro\meter} , 因而推断其属于\ce{CSH}. 同时, 在表格中, 我们发现在表格中主要的元素组成为\ce{Ca}, \ce{Si} 与 \ce{O}, 且\ce{O} 元素成分极大, 推测该区域成分是\ce{CSH} 凝胶. 
\wei{最好对CSH做一些定量的分析, 通过其At\% 来分析. }

\begin{figure}
  \centering
  \includegraphics[width=0.5\textwidth]{assets/15-unpolished-20000x-ETD.png}
  \caption{粉煤灰掺量15\%, 未抛光的硬化水泥石, 在20000x倍率下的ETD图像. }
  \label{fig:ETD-CSH}
\end{figure}

更细致的网络结构如图~\ref{fig:ETD-CSH}所示. 

\subsubsection{水化产物\ce{AFt}}

在ETD图像中发现了如图~\ref{fig:00-05-select}左上方所示的针棒状微观结构, 发现其微观结构规则, 认为其属于晶体; 考虑到其针棒状结构, 推测其属于钙矾石\ce{AFt}. 然而, 遗憾的是, 由于其分布区域过小, 无法选择区域对其做能谱分析, 进行验证. 

\subsubsection{粉煤灰}\label{sec:fa}

\begin{minipage}{\textwidth}
  \begin{minipage}[b]{0.32\textwidth}
    \centering
    \includegraphics[width = \linewidth]{assets/spectrum/15-01-4000x-ETD-FA.png}
    \captionof{figure}{粉煤灰掺量15\%, 01区域的能谱图}
  \end{minipage}
  \hfill
  \begin{minipage}[b]{0.32\textwidth}
    \centering
    \includegraphics[width = \linewidth]{assets/spectrum selection/15-01-04000x-ETD-FA.png}
    \captionof{figure}{粉煤灰掺量15\%, 01区域的ETD图像及选择区域}
    \label{fig:15-01-select}
  \end{minipage}
  \hfill
  \begin{minipage}[b]{0.32\textwidth}
    \centering
    \begin{tabular}{|c|c|c|}
      \hline
      Element & Wt \%  & At \%  \\ \hline
      C K     & 05.53 & 10.11 \\ \hline
      O K     & 29.58 & 40.58 \\ \hline
      MgK     & 00.46 & 00.41 \\ \hline
      AlK     & 26.03 & 21.17 \\ \hline
      SiK     & 30.33 & 23.70 \\ \hline
      CaK     & 05.58 & 03.05 \\ \hline
      FeK     & 02.48 & 00.98 \\ \hline
      \end{tabular}
    \captionof{table}{粉煤灰掺量15\%, 01区域的元素组成}
    \label{tab:15-01}
  \end{minipage}
\end{minipage}

在ETD图像中发现了如图~\ref{fig:15-01-select}的球形的微观结构, 选择以上微观结构区域进行能谱分析, 结果如表~\ref{tab:15-01}所示. 我们发现在表格中主要的元素组成为\ce{Si}, \ce{Al} 与 \ce{O}, 且含有一定量的\ce{C} ; 而粉煤灰主要成分为粉煤灰主要成分是硅铝酸钙, 且钙的含量一般较低, 由于粉煤灰是火电厂的产品, 可能含有\ce{C}, 而且因而认为该区域成分是粉煤灰. 

我们观测到粉煤灰维持了球状的样式, 表明在我们的实验中, 粉煤灰并没有明显地参与反应. 需要说明的是, 这并不意味着粉煤灰在水泥石中不参与任何反应. 实际上, 在水泥石中, 粉煤灰会参与火山灰反应, 而我们实验中的粉煤灰之所以显示出未进行反应的状态, 是由于火山灰反应速率较慢导致的. 即, 在\SI{7}{\day} 我们终止水化时, 几乎可以认为粉煤灰还没有参与反应. 

在宏观层面上, 粉煤灰的掺入会使得水泥石早期的强度降低, 使得后期强度升高. 我们在早期的实验结果(表~\ref{tab:compressive_strength_test})穆庆华的实验结果(表~\ref{tab:strength_fa})有力地表明了这一结论~\cite{mu_fa}. 从表~\ref{tab:strength_fa}中容易发现, 在\SI{28}{\day}前, 掺有粉煤灰的混凝土试块强度都小于不掺粉煤灰的试块的强度, 且强度与粉煤灰掺量呈现负关系; 而在\SI{56}{\day} 及之后, 所有掺粉煤灰的混凝土试块抗压强度均高于不掺粉煤灰的试块的强度, 且强度与粉煤灰掺量呈正关系. 

\begin{table}[!t]
  \centering
  \caption{不同粉煤灰掺量、不同天数的水工混凝土抗压强度~\cite{mu_fa}}
  \begin{tabular}{|c|c|c|ccccc|}
    \hline
    \multirow{2}{*}{编号} & \multirow{2}{*}{粉煤灰细度} & \multirow{2}{*}{粉煤灰掺量~(\unit{\percent})} & \multicolumn{5}{c|}{抗压强度~(\unit{\mega\pascal})}                                                                                            \\ \cline{4-8} 
                        &                        &                        & \multicolumn{1}{c|}{3d}   & \multicolumn{1}{c|}{7d}   & \multicolumn{1}{c|}{28d}  & \multicolumn{1}{c|}{56d}  & 90d  \\ \hline
    A                   & 12.0                   & 0                      & \multicolumn{1}{c|}{18.6} & \multicolumn{1}{c|}{28.4} & \multicolumn{1}{c|}{30.5} & \multicolumn{1}{c|}{31.2} & 31.3 \\ \hline
    B                   & 12.0                   & 10                     & \multicolumn{1}{c|}{17.7} & \multicolumn{1}{c|}{26.5} & \multicolumn{1}{c|}{28.8} & \multicolumn{1}{c|}{32.4} & 33.6 \\ \hline
    C                   & 12.0                   & 20                     & \multicolumn{1}{c|}{16.4} & \multicolumn{1}{c|}{25.7} & \multicolumn{1}{c|}{27.8} & \multicolumn{1}{c|}{33.2} & 33.6 \\ \hline
    D                   & 12.0                   & 30                     & \multicolumn{1}{c|}{15.9} & \multicolumn{1}{c|}{25.3} & \multicolumn{1}{c|}{27.1} & \multicolumn{1}{c|}{33.7} & 34.5 \\ \hline
    E                   & 12.0                   & 40                     & \multicolumn{1}{c|}{14.7} & \multicolumn{1}{c|}{23.6} & \multicolumn{1}{c|}{26.5} & \multicolumn{1}{c|}{34.7} & 35.8 \\ \hline
    \end{tabular}


  \label{tab:strength_fa}
\end{table}


在微观层面上, 我们的实验表明前\SI{7}{\day} 的粉煤灰呈球形, 边界清晰; 随着火山灰反应的进行, 粉煤灰的球形边界会慢慢模糊, 逐渐与周边融为一体. 





\begin{figure}[!t]
  \centering
  \includegraphics[width = 0.5\textwidth]{assets/30-polished-01000x-PMD.png}
  \caption{粉煤灰掺量30\%, 抛光后的硬化水泥块, 在1000x放大倍率下的背散射图. }
  \label{fig:BSE-FA}
\end{figure}

同样地, 我们在 PMD 图像中发现了如图~\ref{fig:BSE-FA}中的圆形微观结构, 根据其形状, 我们判断其为粉煤灰. 容易发现, 其中具有黑色的物质. 由于粉煤灰属于发电的废料, 在其产生的过程中很可能包含碳, 因而推测黑色物质为碳粉. 


