% !TeX root = main.tex

\section{实验步骤}

\subsection{制备掺与不掺粉煤灰的水泥净浆试块}

按照实验方案, 固定水灰比为 \num{0.4}, 粉煤灰掺量采用 \qtylist{0; 15; 30}{\percent} 取代水泥, 制备 3 组水泥净浆试块, 每组 3 组试块, 尺寸为 \qtyproduct{40 x 40 x 40}{\milli\meter}.
试件成型 \SI{24}{\hour} 后拆模, 将其放入标准养护室中养护 \SI{7}{\day} 后测试其抗压强度.

\subsection{通过 SEM 观察水泥净浆试块断面的微观形貌}

将水泥净浆试块在标准养护室养护至 \SI{7}{\day} 之后, 一部分直接取断面观察其微观形貌.
方法如下: 取试块中央部分制成黄豆粒大小, 并在异丙醇中浸泡 \SI{24}{\hour} 以终止水化; 对随后取出置于 \SI{40}{\degreeCelsius} 的真空烘箱中烘干 \SI{24}{\hour}, 之后对样品进行喷金处理, 最后置于 SEM 下观察其微观形貌.

\subsection{观察抛光样品的背散射电子成像}

将水泥净浆试块在标准养护室养护至 \SI{7}{\day} 之后, 取另一部分水泥净浆样品浸泡在异丙醇中以终止水化, 然后在 \SI{40}{\degreeCelsius} 的真空烘箱中烘干 \SI{24}{\hour}, 之后取出将其嵌入环氧树脂中进行抛光处理.
之后对样品进行喷金处理, 最后置于 SEM 下观察其背散射电子成像.
